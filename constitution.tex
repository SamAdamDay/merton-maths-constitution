% The Merton College Mathematics Society Constitution
% ---------------------------------------------------
% The authors of this work hereby place it into the public domain.

% Originally created by Andrew Macarthur and Sam Adam-Day on 27/02/2015
% Edited by Daniel Bregman 2015-03-02

\documentclass[pdftex,a4paper]{report}
\usepackage[hidelinks]{hyperref}
\usepackage{enumitem}
\usepackage{amsmath}


\title{Merton College Mathematics Society Constitution}


\begin{document}


\begin{center}

	{\large Merton College Mathematics Society} \\[5pt]
	{\Huge Constitution}

\end{center}


\section*{Definition and Membership}

\begin{enumerate}[label=\Roman*)]
	\item The name of the body hereby constituted shall be the Merton College Mathematics Society, hereinafter referred to as `the Maths Society'.
	\item The Maths Society will be an institution, governed by in accordance with the Constitution and Standing Orders of the Merton College Junior Common Room (hereinafter referred to as `the JCR').
	\item The purpose of the Maths Society shall be to protect and further the academic interests of students of mathematics and to foster and maintain a vibrant and active mathematical community within the College.
	\item The members of the Maths Society shall be all those members of the JCR studying for one of the following types of undergraduate degree:
	\begin{enumerate}[label=\roman*)]
		\item Single Honours Mathematics, or Joint Schools including Mathematics,
		\item Single Honours Computer Science, or Joint Schools including Computer Science,
	\end{enumerate}
	except those who have opted out the Maths Society.
	\item Membership of the Maths Society shall confer the following rights upon an individual:
	\begin{enumerate}[label=\roman*)]
		\item The right to call, attend, speak and vote at all Meetings of the Maths Society.
		\item The right to vote at the elections of the Maths Society.
		\item The right to stand in any of the Maths Society's elections, subject to any additional conditions of eligibility.
		\item The right to enjoy the facilities and events provided by the Maths Society.
	\end{enumerate}
	\item All Members of the Maths Society shall have the right to opt out of this Membership by informing the Supremum or a Meeting of the Maths Society. Any undergraduate normally eligible for membership who has opted out may automatically reinstate their membership by application to the Supremum or a Meeting of the Maths Society.
	\item The General Meeting of the JCR shall have the same powers as a Meeting of the Maths Society, and shall take precedence if the two conflict. 

\end{enumerate}


\section*{The Supremum}

\begin{enumerate}[resume*]
	\item One member of the Maths Society shall hold the office of Supremum of the Maths Society.
	\item The Supremum shall perform the r\^ole of Subject President for Mathematics as definted in the JCR Standing Orders.
	\item The Supremum shall be appointed through a free and fair election from the members of the Maths Society, and shall normally hold office for no longer than three terms.
	\item The Supremum shall be responsible for the administration of the Maths Society and its finances, the organisation of the Maths Society Dinner and of the Maths Society Garden Party, as well as other such duties as are reasonably required of the post.
	\item The Supremum may resign their office by informing the entire membership of the Maths Society.
\end{enumerate}


\section*{Meetings}

\begin{enumerate}[resume*]
	\item There shall be at least one Meeting of the Maths Society held each academic year.
	\item Any member of the Maths Society shall have the right to call a Meeting of the Maths Society in accordance with the Standing Orders.
	\item A $\frac{2}{3}$ majority vote on the matter shall be a necessary and sufficient condition for the amendment of the Constitution.
	\item A simple majority vote on the matter shall be a necessary and sufficient condition for the amendment of the Standing Orders.
	\item A simple majority vote on the matter shall be a necessary and sufficient condition for the dismissal of the Supremum.
\end{enumerate}


\section*{The Constitution and Standing Orders}

\begin{enumerate}[resume*]
	\item The Standing Orders shall supplement and not overrule the Constitution. In the event of any conflict between the Standing Orders and the Constitution, the Constitution shall take precedence.
	\item The Supremum shall be the initial interpreter of the Constitution and Standing Orders. In the event of any dispute over this interpretation, the ruling of a Meeting of the Maths Society shall be determinant.
	\item All previous Constitutions of the Merton College Mathematics Society are hereby expressly revoked. This Constitution shall have effect from 2nd March 2015.
\end{enumerate}


\end{document}
