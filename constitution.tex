% The Merton College Mathematics Society Constitution
% ---------------------------------------------------
% The authors of this work hereby place it into the public domain.

% Originally created by Andrew Macarthur and Sam Adam-Day on 27/02/2015


\documentclass[pdftex,a4paper]{report}
\usepackage[hidelinks]{hyperref}
\usepackage{enumitem}
\usepackage{amsmath}


\title{Merton College Mathematics Society Constitution}


\begin{document}


\begin{center}

	{\large Merton College Mathematics Society} \\[5pt]
	{\Huge Constitution}

\end{center}


\section*{Definition and Membership}

\begin{enumerate}[label=\Roman*)]
	\item The name of the body hereby constituted shall be Merton College Mathematics Society, hereinafter referred to as `the Maths Society'.
	\item The Maths Society will be an institution, governed by in accordance with the Constitution and Standing Orders of the Merton College Junior Common Room (hereinafter referred to as `the JCR').
	\item The purpose of the Maths Society shall be to protect and further the academic interests of students of mathematics and to foster and maintain a vibrant and active mathematical community within the College.
	\item The members of the Maths Society shall be all those members of the JCR studying for one of the following types of undergraduate degree:
	\begin{enumerate}[label=\roman*)]
		\item Mathematics (Single Honours, or Joint Schools),
		\item Computer Science (Single Honours, or Joint Schools),
	\end{enumerate}
	except those who have opted out the Maths Society.
	%\begin{equation*}
		%\begin{split}
	 	%\{x \in JCR \mid x\text{ undergraduate }\wedge (x\text{ studying for Single Honours Mathematics degree }\vee\text{ x studying for Joint Schools Mathematics degree})\} \backslash \{x \in JCR \mid x\text{ has opted out of the Maths Society}\}
	 	%\end{split}
	%\end{equation*}
	\item Membership of the Maths Society shall confer the following rights upon an individual:
	\begin{enumerate}[label=\roman*)]
		\item The right to call, attend, speak and vote at all Meetings of the Maths Society.
		\item The right to vote at the elections of the Maths Society.
		\item The right of candidature for the presidency of the Maths Society, subject to satisfying any conditions of eligibility.
		\item The right to enjoy the facilities and events provided by the Maths Society.
	\end{enumerate}
	\item All Members of the Maths Society shall have the right to opt out of this Membership.
	\item The Maths Society shall not discriminate against any person on the grounds of their race, religion, nationality, gender, sexual orientation or disability.
	\item The General Meeting of the JCR shall be the supreme governing body of the Maths Society.
	\item Any Maths Society Member shall have the right to complain formally about any aspect of the running of the Maths Society or the actions of any person involved in the administration of the Maths Society insofar as those actions relate to their duties and responsibilities to the Maths Society.

\end{enumerate}


\section*{The President}

\begin{enumerate}[label=\Roman*)]
	\item The Maths Society President shall be one and the same person as the `Subject President' for Mathematics as defined in the JCR Standing Orders.
	\item The Maths Society President shall be referred to as `the Supremum'.
	\item The Supremum shall be elected freely and fairly by the Maths Society Membership from within their own number and is entrusted with the administration of Maths Society affairs and facilities. The Supremum when acting ex officio shall be bound by the Constitution and Standing Orders of the Maths Society.
	\item The Supremum shall be responsible for the organisation of the Maths Society Dinner and the Maths Society Garden Party, and for the management of the finances of the Maths Society. The Supremum shall also undertake such duties as are reasonably required of the post.
	\item The Supremum shall be the sole interpreter of the Constitution and Standing Orders. In the event of any dispute over this interpretation, the ruling of a Meeting of the Maths Society shall be determinant.
\end{enumerate}


\section*{Meetings}

\begin{enumerate}[label=\Roman*)]
	\item Any member of the Maths Society shall have the right to call a Meeting of the Maths Society, in accordance with the Standing Orders.
	\item A $\frac{2}{3}$ majority vote on the matter shall be a necessary and sufficient condition for the amendment of the Constitution.
	\item A simple majority vote on the matter shall be a necessary and sufficient condition for the amendment of the Standing Orders.
	\item A simple majority vote on the matter shall be a necessary and sufficient condition for the dismissal of the Supremum.
\end{enumerate}


\section*{The Constitution and Standing Orders}

\begin{enumerate}[label=\Roman*)]
	\item The Standing Orders shall supplement and not overrule the Constitution. In the event of any conflict between the Standing Orders and the Constitution, the Constitution shall take precedence.
	\item All previous Constitutions of the Merton College Mathematics Society are hereby expressly revoked. This Constitution shall have effect from 2nd March 2015.
\end{enumerate}


\end{document}