% The Merton College Mathematics Society Standing Orders
% ------------------------------------------------------
% The authors of this work hereby place it into the public domain.

% Originally created by Andrew Macarthur and Sam Adam-Day on 27/02/2015
% Edited Daniel Bregman 2015-03-02


\documentclass[pdftex,a4paper]{report}
\usepackage[hidelinks]{hyperref}
\usepackage{enumitem}
\usepackage{amsmath}


\title{Merton College Mathematics Society Standing Orders}


\begin{document}


\begin{center}

	{\large Merton College Mathematics Society} \\[5pt]
	{\Huge Standing Orders}

\end{center}

These Standing Orders shall import the definitions and notions of the Constitution of the Merton College Mathematics Society (`the Maths Society').


\section*{General}

\begin{enumerate}[label=\arabic*)]
	\item The Supremum shall maintain an up-to-date Master Copy of the Constitution and Standing Orders, which shall be available for the consultation by the members of the Maths Society.
	\item These Standing Orders shall supplement and not overrule the Constitution. In the event of any conflict between the Standing Orders and the Constitution, the Constitution shall take precedence.
\end{enumerate}


\section*{Election of the Supremum}

\begin{enumerate}[resume*]
	\item Elections for the post of Supremum shall be conducted in accordance with the JCR Standing Orders, which shall take precedence in the event of conflict with this section.
	\item The election shall be conducted in accordance with the procedure outlined below, with the incumbent Supremum acting as Returning Officer:
	\begin{enumerate}[label=\roman*)]
		\item The Returning Officer shall call for candidates for election no later than one month before the Maths Society Dinner.
		\item The Returning Officer shall open the voting no later than one week before the Maths Society Dinner.
		\item The Returning Officer shall close the voting at 12 noon on the day of the Maths Society Dinner.
		\item The Supremum-Elect shall take office on the day immediately following the Maths Society Dinner and shall remain in office until succeeded (approximately one academic year later, immediately following the next Maths Society Dinner).
	\end{enumerate}
	\item The following conditions shall also apply to the election:
	\begin{enumerate}[label=\roman*)]
		\item Candidates may withdraw from the election at any point before the voting has opened.
		\item Voting shall be by the system of a single transferable vote.
		\item The ballot shall list all candidates in alphabetical order of their surname, as well as the option to Re-Open Nominations (RON), which shall appear last.
		\item Any indication of a vote for a candidate expressed, where it is the only indication so expressed, shall be deemed as a first preference vote for the aforementioned candidate.
	\end{enumerate}
	\item In the event that no Supremum can be elected in a given year (for example in the case of a tie or the election of RON), the Returning Officer shall consult with the Returning Officer of the JCR and normally hold a by-election.
\end{enumerate}


\section*{Duties of the Supremum}

\begin{enumerate}[resume*]
	\item The Supremum shall fulfil all duties reasonably required by the JCR in relation to Freshers' Week. In particular, they shall hold in Freshers' Week a gathering of all members of the Maths Society, especially including all Merton Freshers who might reasonably expect to qualify for membership of the Maths Society within the current academic year (i.e. those undergraduate students joining Merton College to study for an appropriate degree).
	\item The Supremum shall organise a Maths Society subject dinner, to be held once per academic year in Hilary Full Term. The Supremum should invite to this dinner all students and fellows whose subject or field of study is mathematics or computer science, as well as all those involved or recently involved in teaching mathematics or computer science at Merton College.
	\item The Supremum shall organise a garden party, to be held once per academic year in Trinity Full Term, with the same scope of invitations as for the subject dinner.
	\item The Supremum shall have responsibility for the finances of the Maths Society. To this end, the Supremum shall keep records of all relevant transactions.
	\item At the end of their term in office, the Supremum shall submit the aforementioned records to their successor who shall then conduct the audit.
	\item The Supremum shall perform all other duties reasonably required of them by virtue of their position in the Maths Society and as required by the JCR Constitution and Standing Orders.
\end{enumerate}


\section*{Meetings}

\begin{enumerate}[resume*]
	\item In order to call a Meeting, a member shall notify all other members of the Maths Society of their intention to call a Meeting, and of the purpose of the Meeting, no later than one week before the date of the Meeting.
	\item All Meetings shall take place in Full Term immediately following a General Meeting of the JCR.
	\item A Meeting shall be quorate if and only if the number of members of the Maths Society present at the Meeting is at least $2$, and at least $\frac{1}{4}$ of the total number of members of the Maths Society.
	\item A quorum count must be held to establish that a Meeting is quorate.
	\item If the Meeting is found to be inquorate, it shall adjourn for ten minutes; if after reconvening it remains inquorate then this shall be recorded and the Meeting shall be abandoned.
	\item At the beginning of a Meeting the participants shall appoint from within their own number a Chair and a Secretary. The Chair and the Secretary shall not be the same person.
	\item The Chair, who shall normally be the Supremum, shall be responsible for the smooth running of the Meeting, maintaining order, preserving an atmosphere of consideration, and ensuring that the Constitution and Standing Orders are respected.
	\item The Secretary shall record minutes of the business transacted at the Meeting. These minutes shall be made available to the members of the Maths Society shortly after the Meetingn, and maintained by the Supremum.
	\item Voting within a Meeting shall ordinarily be by show of hands, coordinated by the Chair, with the assistance of the Secretary. In the case of a tie the Chair shall have the casting vote.
	\item All Meetings shall have the powers set out in the Constitution, namely the power to amend the Constitution and Standing Orders, and to dismiss the Supremum.
\end{enumerate}


\section*{By-Elections}

\begin{enumerate}[resume*]
	\item Should the Supremum be dismissed by a Meetingn or resign, a by-election shall be triggered.
	\item A Meeting at which the Supremum is dismissed shall appoint a Returning Officer for the by-election. If the Supremum resigns they shall normally act as Returning Officer for the by-election; otherwise a meeting shall be called at which a Returning Officer shall be appointed.
	\item The procedure to be followed for the by-election shall follow the normal election procedures except that references to the Maths Society Dinner should instead refer to a date selected by the Returning Officer no later than two weeks after the triggering of the by-election.
\end{enumerate}


\end{document}